\subsection{Computation on real numbers}
\label{section:function}

\subsubsection{\texorpdfstring{$\classL$}{L} computable real functions}

\begin{example}
 The binary addition and the binary multiplication are
 $([\rhoR, \rhoR], \rhoR)$-$\classFLtwo$-computable.
\end{example}

\begin{lemma}
 The exponential function on restricted to $[0,1]$
 is $(\rhoRunit, \rhoR)$-$\classFLtwo$-computable.
\end{lemma}
Since the Maclaurin series $\exp(t) = \sum_n x^n / n!$ converges fast enough,
$2^{-m}$-approximation of $\exp(t)$ for $t \in [0,1]$ is computable 
by polynomial many multiplications, additions, and division on rationals.
Since iterated multiplications, iterated additions, division on integers
are in $\classFL$ \cite{chiu2001division},
$\exp$ is $(\rhoRunit, \rhoR)$-$\classFLtwo$-computable.
On the other hand, without restriction on the domain, $\exp$ is not 
$(\rhoR, \rhoR)$-$\classFPtwo$ computable, 
and is not $(\rhoR, \rhoR)$-$\classFLtwo$ computable.



\begin{lemma}
  The sine function $\sin \colon \R \to \R$ is
 is $(\rhoR, \rhoR)$-$\classFLtwo$-computable.
\end{lemma}
Like $\exp$, the Maclaurin series of $\sin$ converges fast enough,
so it is obvious that $\sin$ is log-space computable if it is restricted to $[-4, 4]$.
Computing $x \in [-4, 4]$ satisfying $x = t + 2n\pi$ for some $n \in \N$ is
$(\rhoR, \rhoR|^{[-4,4]})$-$\classFLtwo$-computable because division is log-space computable and $\pi$ is $\rhoR$-$\classFL$-computable from the following series:
\begin{equation}
 \pi = \sum_{k=0}^\infty \frac{1}{16^k} 
  \left( \frac{4}{8k+1} - \frac{2}{8k+4} - \frac{1}{8k+5} - \frac{1}{8k+6} \right).
\end{equation}
Hence, $\sin$ is $(\rhoR, \rhoR)$-$\classFLtwo$-computable.



\subsubsection{Finding polynomial roots is \texorpdfstring{$\classFNCtwo$}{FNC}-computable}

\paragraph{Representation of multisets of complex numbers}

For two multisets of $n$ complex numbers $S = \{z_1, \dots, z_n\}$,$S' = \{z'_1, \dots, z'_n\}$,
we define the distance between $S$ and $S'$ as
$d(S, S') = \min_{\sigma \in \mathrm{Sym}(n)} \max_{j \le n}|z_j - z'_{\sigma(j)}|$,
where $\mathrm{Sym}(n)$ is the set of all permutations of $n$ elements.

We represent multisets of complex numbers as converging sequences of 
multisets of rational complex numbers.
Define the representation $\rhoMSet$ as follows:
Let $S$ be a multiset of $n$ complex numbers,
$\phi \in \LM$ is a $\rhoMSet$ name of $S$ if
$\phi(0^m) = ( a_1, b_1, a_2, b_2, \dots, a_n, b_n ) \in (\D_m)^{2n}$
for all $m \in \N$.
Let $S_m = \{a_1+b_1 i, \dots, a_n+b_n i\}$, 
$d(S, S_m) \le 2^{-m}$.

We can also represent multisets $S$ as vectors $z \in \C^n$
such that $S = \{z\}$.
More formally we define representation $\rhoMSet'$ as
$\rhoMSet'(\phi) = \{z\}$ if $\rho_{\C^*}(\phi) = (z_1, \dots, z_n)$.
It is easy to see that $\rhoMSet' \le_\classFLtwo \rhoMSet$,
but the following lemma says they are not equivalent even under computability.
\begin{lemma}
 Identity function on multisets is not $(\rhoMSet, \rhoMSet')$-computable.
\end{lemma}
This means that $\rhoMSet'$ is more informative than $\rhoMSet$.



\paragraph{Finding polynomial roots is $\classFNCtwo$-computable}

We write $P_n$ for the set of degree $n$ monic 
polynomials (i.e. the leading coefficient is equal to $1$) with real coefficients.
We write $\OpPolyRoot(p)$ for the multiset of all the roots of $p \in P_n$, that is,
\begin{equation}
 \OpPolyRoot(p) \coloneqq \{ z_1, \dots, z_n \} \text{ s.t. } p(x) = \prod_j (x - z_j).
\end{equation}

We define the representation $\rho_{P_n}$ of degree $n$ monic polynomials as 
$\phi \in \LM$ is a $\rho_{P_n}$ name of $p \in P_n$ 
if $\phi$ is a $\rho_{\R^n}$ name of $\langle a_{n-1}, \dots, a_0 \rangle$ 
such that $p(x) = x^n + a_{n-1}x^{n-1} + \cdots a_0$.
Define the representation $\rho_{P_*}$ of monic polynomials as
$p(x) = \rho_{P_*}(\langle 0^n, \phi \rangle) \iff p(x) = \rho_{P_n}(\phi)$.


\begin{theorem}
 \label{theorem:finding-roots-is-in-NC}
 $\OpPolyRoot$ is $(\rho_{P_*}, \rhoMSet)$-$\classFNC$-computable.
\end{theorem}

\begin{corollary}
 $\OpPolyRoot$ is $(\rho_{P_n}, \rhoMSet)$-$\classFNC$-computable
\end{corollary}


\begin{proof}
Neff showed that approximating the roots of polynomial with integer coefficients
is in $\classFNC$.
\begin{theorem}
[NC polynomial root isolation \cite{neff1994specified}]\label{theorem:neff1994}
Give and algorithm $A$ such that,
for any choice of integers $n$, $m$, $k$, and a polynomial $p \in P_n$
s.t. $|p| \le 2^m$,
$A$ computes $k$-digit approximation to the roots of $p$ 
in at most $C \log^e(n + m + k)$ parallel steps, 
using at most $D(n + m + k)^f$ processors, where $C, D, e, f$ are positive
constants which are independent of $n$, $m$, and $k$.
\end{theorem}
The \emph{size} of a polynomial, denoted $|p|$, is the maximum absolute value
 of coefficients in $p$, and the \emph{distance} of
two polynomial $p, q \in P_n$, denoted $d(p, q)$, is $|p-q|$.
The Theorem~\ref{theorem:finding-roots-is-in-NC} follows immediately
Theorem~\ref{theorem:neff1994}
if there is a polynomial $\mu$ such that for all $p, q \in P_n$ and
$k, m \in \N$ satisfying $|p| \le 2^m$,
\begin{equation}
  d(p, q) \le 2^{-\mu(k, m, n)} \Rightarrow d(\OpPolyRoot(p), \OpPolyRoot(q)) \le 2^{-k},
\end{equation}
by following algorithm:
Given $k \in \N$ and a $\rho_{P_n}$ name of $p \in P_n$, 
compute $m \in \N$ and polynomial $q \in {P_n}$ with rational coefficients
satisfying $|p| \le 2^m$ and $d(p, q) \le 2^{-\mu(k, m, n)}$,
then find all the roots of $2^{\mu(k, m, n)} \cdot q(x)$ by the algorithm $A$ 
in Theorem~\ref{theorem:neff1994}.


Here we use Mosier's analysis to show the existence of such a polynomial $\mu$
\cite{mosier1986root}.

Define the set $Z(p, \epsilon)$ of complex numbers as
\begin{equation}
 Z(p, \epsilon)
 = 
  \{z \mid \exists q \in P_n \text{ s.t. } q(z) = 0 \wedge d(p, q) \le \epsilon\},
\end{equation}
for any polynomial $p \in P_n$ and $0 < \epsilon < 1$.
First we show that for any $u \in Z(p, \epsilon)$, there is $v \in \OpPolyRoot(p)$ such that $|u - v| \le 2^{m+2}\sqrt[n]{\epsilon}$.
\begin{theorem} 
[{\cite[Theorem 1]{mosier1986root}}]
\label{theorem: root neighborhoods 1}
 Given $p(x) \in P_n, then$
 \begin{equation}
  Z(p, \epsilon) = \left\{u \;\Big{|}\;  \left| p(u) / \Sigma_{j=0}^n |u|^j \right| \le \epsilon \right\}
 \end{equation}
\end{theorem}
For any $u \in \C$, if $v$ is the root of $p$ such that 
$|u - v| \le |u - v'|$ for all roots $v'$ of $p$,
\begin{equation}
 |v - u|^n
 \le \prod_{v \in \OpPolyRoot(p)} |v_j - u|
 = |p(u)|.
\end{equation}
It is follows from  Rouch\'a's theorem that
any root $v$ of a monic polynomial $p \in P_n$ satisfies that
$|v| \le 1 + \max\{a_{n-1}, \dots, a_0\} = 1 + |p|$.
Since there is a polynomial $q$ such that $q(u)=0$ and $d(p, q) \le \epsilon$,
$|u| \le 1+|q| \le 1 + |p| + d(p, q) \le 2^m + 2$ so
\begin{equation}
 |v - u|
 \le
 \sqrt[n]{|p(u)|}
 \le
 \sqrt[n]{\epsilon \sum_{j=0}^n |u|^j}
 \le
 \sqrt[n]{\epsilon \cdot (|u|+1)^n}
 \le
 2^{m+2} \sqrt[n]{\epsilon}.
\end{equation}


Let $\epsilon' = 2^{m+2} \sqrt[n]{\epsilon}$,
\begin{equation}
 Z(p, \epsilon) \subseteq \bigcup_{v \in \OpPolyRoot(p)} D(v, \epsilon')
\end{equation}
where $D(v, \epsilon') = \{u \mid |u - v| \le \epsilon'\}$.
Since the diameter of each connected component of the right side 
is $2n\epsilon$ at most, $|u - u'| \le 2n\epsilon$ for all 
$u, u' \in Z(p, \epsilon)$ if $u$ and $u'$ are in the same connected component
of $Z(p, \epsilon)$.

Let $\mu(n, k, m) = n(k+m+n+2)$.
We show that 
$
d(p, q) \le 2^{-\mu(n, k, m)} \Rightarrow d(\OpPolyRoot(p), \OpPolyRoot(q)) \le 2^{-k}
$
for all $q \in P_n$.
Let $\epsilon \in (0,1)$, $q$ be a polynomial satisfying that
 $d(p, q) \le \epsilon$.
We assume that the (ordered) roots $v_1, \dots, v_n$ of $p$
and the (ordered) roots $u_1, \dots, u_n$ of $q$ satisfy that
$v_j$ and $u_j$ are in the same connected component of $Z(p, \epsilon)$.
Following theorem says that there is some ordering of roots satisfies this assumption.
\begin{theorem}
[\cite{mosier1986root}]
\label{theorem: root neighborhoods 2}
 For any $0 < \epsilon < 1$, if $q(z) \in N(p, \epsilon)$,
 then $q(z)$ and $p(z)$ have the same number of roots,
 counting multiplicities, in each connected component of $Z(p, \epsilon)$.
\end{theorem}
Let $\epsilon = 2^{-\mu(n,m,k)}$ then for all $1 \le j \le n$
\begin{equation}
 |u_j - v_j| 
 \le
 2n\epsilon'
 \le
 2n \cdot 2^{m+2} \cdot 2^{-(k+m+n+2)} 
 \le
 2^{-k},
\end{equation}
hence
$
 d(\OpPolyRoot(p), \OpPolyRoot(q)) \le 2^{-k}.
$
\end{proof}