\subsection{Second-order polynomial}
\label{section:TTF}

This section reviews the extended framework of Computable Analysis~%
\cite{kawamura2012complexity}.
A (total) function $\phi \colon \Sigma^* \to \Sigma^*$ is \emph{length monotone}
if  $|\phi(u)| \le |\phi(v)|$ whenever $|u| \le |v|$.
The set of length monotone functions is denoted $\LM$.
We write $\Pred$ for the set of functions $\phi \colon \Sigma^* \to \{0, 1\}$.
Since $|\phi(u)| = |\phi(v)| = 1$ for all $\phi \in \Pred$ and $u, v \in \Sigma^*$,
$\Pred$ is a subset of $\LM$.

\begin{definition}
 A machine $M$ computes a multi-function $A \colon \LM \rightrightarrows \LM$ if for any
 $\phi \in \dom A$, there is $\psi \in A[\phi]$ such that $M^\phi(u) = \psi(u)$ for all $u \in \Sigma^*$.
\end{definition}

The \emph{size} of a length monotone $\phi$, denoted $|\phi|$,
is a (non-decreasing) function from $\N$ to $\N$ satisfying 
$|\phi|(|u|) = |\phi(u)|$.
This is well-defined since a length monotone function maps 
all $n$-length strings to $m$-length strings.

\emph{Second-order polynomial} in type-1 variable $L \colon \N \to \N$
and type-0 variable $n \in \N$ 
defined inductively as follows:
a positive integer is a second-order polynomial;
the variable $n$ is a second-order polynomial;
$P+Q$, $P \cdot Q$ and $L(P)$ are
second-order polynomials if $P$ and $Q$ are second-order polynomials.
Note that if $P$ is a second-order polynomial and $L$ is a (usual) first-order
polynomial, then $P(L)$ is a first-order polynomial.

\begin{definition}
\begin{enumerate}
 \item We write $\classFPtwo$ (resp. $\classFPSPACEtwo$) for the class of
       multi-functions from $\LM$ to $\LM$ computed by a machine that runs
       in second-order polynomial time (resp. space).
 \item We write $\classPtwo$ (resp. $\classPSPACEtwo$) for the class of
       multi-functions from $\LM$ to $\Pred$ computed by a machine that runs
       in second-order polynomial time (resp. space).
\end{enumerate}
\end{definition}

\begin{lemma}
 \begin{enumerate}
  \item Functions in $\classFPtwo$ (resp. $\classFPSPACEtwo$) map 
	length-monotone functions in $\classFP$ to $\classFP$ 
	(resp. $\classFPSPACE$).
  \item Functions in $\classPtwo$ (resp. $\classPSPACEtwo$) map 
	length-monotone functions in $\classFP$ to $\classP$
	(resp. $\classPSPACE$).
 \end{enumerate}
\end{lemma}








