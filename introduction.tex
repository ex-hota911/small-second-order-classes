%#!platex fulltext.tex

\subsection{Computable analysis}

We investigate the computational complexity of several problems in 
numerical analysis.
The problems we study in this thesis are of the following kind:
How does smoothness change the complexity to solve differentiable equations?
How much resource is needed to compute the roots of a polynomial?
Which operators on real function are feasibly computable but inherently sequential?
In this thesis, we use the model of the real computation from the field of 
Computational Analysis.

\emph{Computable Analysis} 
\cite{ko1991complexity,weihrauch00:_comput_analy}
studies real functions that are computable approximately 
by digital (finite-precision floating-point) machines.
Here, polynomial-time computability and other notions of complexity 
is from this field and measure how hard it is to approximate real functions
with specified precision  (Sect.~\ref{section:TTE}). 

In the theory of \emph{real computation}, 
there is no ``Church-Turing thesis''.
Church-Turing thesis states that a function on finite strings or natural 
numbers is ``effectively'' or algorithmically computable 
if and only if some Turing machine can compute it.
Many other computation models such as $\lambda$-calculus or recursion
were proposed and shown to be equivalent to Turing machine.
In contrast, there are models about computation on real numbers, 
and some of them are known not to be equivalent to others.
Another model is the Blum-Shub-Smale (BSS) machine \cite{blum1988theory}.
The main difference between BSS machines and machines in Computable Analysis is
that a BSS machine has registers containing real numbers with infinite precision
while digital machine can work on finite-precision number.
Because of this difference,
machines in Computable Analysis cannot compute noncontinuous functions
but the algorithm described in Computable Analysis 
can be used as numerical computation in real computers
 while we do not know
how to realize BSS machines.




\subsection{Type-two complexity theory for small classes}
The complexity of an operation $F$ on real functions
is stated indirectly in the non-uniform form like
\begin{quote}
 if $f$ is in the complexity class $\classonefont X$,
 then $F(f)$ is in the complexity class $\classonefont Y$, and \\
 there is $f$ in $\classonefont X$ such that $F(f)$ is hard for
 the complexity class $\classonefont Z$.
\end{quote}
That is, there is some restriction to apply the framework TTE
to complexity theory.
There is a widely-recognized definition of polynomial-time computable 
real functions whose domain is compact.
However, there is not such a model for
operators $F$ on real functions in purpose to state
directly that $F$ is polynomial-time computable or hard for $\classPSPACE$.

Kawamura proposed an extension of this framework \cite{kawamura2012complexity}.
The key idea is representing uncountable 
objects by \emph{length-monotone} string functions, which make it possible to define the
\emph{size} of function and to bound resources by size of input.
In that framework we can state directly about the complexity of
 an operator $F$ in the effective or uniform form like
\begin{quote}
 $F$ is in the complexity class $\mathcal Y$, and \\
 $F$ is hard for $\mathcal Z$ under the $\mathcal X$ reduction,
\end{quote}
where $\mathcal{X, Y, Z}$ are type-two complexity classes analogous to $\classonefont{X, Y, Z}$.
There is no obvious way to define consistent type-two classes 
analogous to usual (type-one) complexity classes.

First, we introduce new small type-two classes.
Type-two analogues of the classes $\classP$, $\classNP$, and $\classPSPACE$
are defined by bounding resource of oracle machines by using 
second-order polynomials on the size of the input string functions,
\cite{kawamura2012complexity}.
Using second-order logarithmic function,
we define type-two classes analogous to the logarithmic-space 
complexity class $\classL$ and the circuit complexity class $\classNC$, 
which are both contained in $\classP$.
While there are various models of relativized logarithmic space,
we choose the \emph{constant stack machine} \cite{aehlig2007relativizing} 
since it is consistent with relativized circuit complexity classes 
and it makes some elemental operation log-space computable.
We also define $\classP$-completeness under log-space reductions.

Informally speaking, operators in type-two $\classL$ have
memory friendly algorithm and operators in type-two $\classNC$
are parallelizable.
If an operation is $\classP$-complete,
it is feasibly computable but inherently sequential. 

Then, we apply this framework to several problems in analysis.
Our first application is the problem of finding roots of a polynomial 
from its coefficients.
In numerical analysis, it is known to be generally ill-conditioned, 
that is, the roots are sensitive to 
small errors in the coefficients \cite{wilkinson1963rounding}.
However, with the complex analysis by Mosier \cite{mosier1986root},
we can bound the precision needed to compute
roots by a polynomial in the required precision of the solution.
We combine this analysis with the discrete result 
that the problem to approximate the roots of the polynomial with integer 
coefficients is $\classNC$ computable \cite{neff1994specified}
to show that roots of a polynomial with real coefficients
is $\classNC$ computable.
Second application is about $\classP$-complete operators.
There exist a few result about $\classP$-completeness of problems of numerical analysis:
The inverse function $f^{-1}$ can be $\classP$-complete under the log-space
reduction even if $f$ is log-space computable \cite{ko1983computational};
The fix-points of contracting mappings can be $\classP$-complete under the
$\classNC$ reduction even if contracting mappings are all $\classNC$ computable \cite{hoover1991real}.
We express these theorems in constructive way and
show type-two $\classP$-completeness of these operators.




%Chap.~\ref{chapter:computable-analysis} introduces Type-two Theory of Effectivity, the framework of Computable Analysis.
%We review basic concepts of TTE in Sect.~\ref{section:TTE} and
%Kawamura's extended framework for TTE in Sect.~\ref{section:TTF}.
%In Sect.~\ref{section:small-classes}, we introduce our new type-two classes
%$\classLtwo$ and $\classNCtwo$.
%We also define $\classPtwo$-completeness under reductions using these small classes.
%In Chap.~\ref{chapter:applications}, we investigate the complexity of some analytic problems.
%In Sect.~\ref{section:function}, we prove that finding roots of a polynomial 
%from coefficients is $\classNC$ computable.
%In Sect.~\ref{section:P-complete}, we show the $\classP$-completeness of 
%the inverse operation and the fix-point operation.
%In Sect.~\ref{section:differentiable}, we study the complexity of
%smooth differential equations.
%Most part in that section only depends on Sect.~\ref{section:TTE}
%since we mainly discuss in non-uniform way.
%We summarize this thesis and discuss about future works in Chap.~\ref{chapter:conclusion}.


\paragraph{Notation}
Let $\N$, $\Z$, $\Q$, $\R$ denote the set of natural numbers,
integers,
rational numbers and 
real numbers, respectively.

We assume that any polynomial is increasing,
since it does not change the meaning of 
polynomial-time computable or polynomial-space computable.

