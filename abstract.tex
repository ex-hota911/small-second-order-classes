\begin{abstract}
Computable Analysis is a branch of computability theory that
studies real numbers or real functions computable on digital machines.
We use the framework of Type-two Theory of Effectivity proposed by Weihrauch,
which is based on the models by Turing, Grzegorczyk and Lacombe.
Not only computability, but also time and space complexity of real functions, 
has been studied since Ko's work in the 1980s. 
In this field,
we analyze the complexity of algorithms in numerical analysis 
and discuss lower bounds on the speed of reliable numerical algorithms.

The first topic of this thesis is the lower bounds of the complexity of the solution~$h$
for the ordinary differential equation of the form 
$h(0)=0$, $h'(t) = g(t, h(t))$.
It is well-known that this equation has a unique solution if $g$ is Lipschitz continuous.
Kawamura showed in 2010 that the solution~$h$ can be $\classPSPACE$-hard
even if $g$ is Lipschitz continuous and polynomial-time computable. 


We place further requirements on the smoothness of $g$ 
and obtain the following results: 
the solution~$h$ can still be $\classPSPACE$-hard
if $g$ is assumed to be of class $\classC ^1$; 
for each $k \geq 2$, 
the solution~$h$ can be $\classCH$-hard
if $g$ is of class $\classC ^k$,
where $\classCH$ (Counting Hierarchy) is a complexity class contained in $\classPSPACE$
and containing $\classNP$, $\classPP$, and $\classPH$.
This result implies that the smoothness alone does not help to 
compute the solution $h$ significantly.


The second topic is the complexity of problems in numerical analysis 
that are known to be polynomial-time computable.
Kawamura and Cook proposed an extended framework for TTE.
They define type-two complexity classes analogous to $\classP$, $\classNP$,
and $\classPSPACE$ by bounding time or space by the second-order polynomial
in the size of the input.
In this thesis, we focus on small classes
and define type-two analogues of the log-space class $\classL$ and
the circuit class $\classNC$.
We also define the type-two $\classP$-completeness under suitable log-space reductions.

We apply this framework to some specific problems.
First, we study the complexity of the problem to find the roots of
the polynomial from its coefficients.
Using Mosier's analysis, we can bound the required input precision 
by a polynomial in the precision of the solution.
We combine this analysis and the discrete result 
that the problem to approximate the roots of the polynomial with integer 
coefficients is $\classNC$ computable,
and we show that the polynomial roots are type-two $\classNC$ computable.
Second, we give two examples of $\classP$-complete operators,
the inverse operation and the fix-point operation on contracting mappings.
It is known that these operators are $\classP$-complete in a non-uniform sense:
the fix-points of $\classNC$-computable contracting mappings,
as well as the inverse function $f^{-1}$ of a one-to-one $\classL$-computable function $f$,
can be $\classP$-complete.
We express these theorems in the uniform way and
show that these operators themselves are type-two $\classP$-complete.
\end{abstract}

